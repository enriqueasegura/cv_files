% LaTeX resume using res.cls
\documentclass[margin,line]{res}

\usepackage[utf8]{inputenc} 
\usepackage[T1]{fontenc}
\usepackage{ebgaramond}


\RequirePackage{filecontents}
\usepackage[colorlinks = true,
linkcolor = blue,
urlcolor  = blue,
citecolor = blue,
anchorcolor = blue]{hyperref}

%\lfoot{\hspace{-\sectionwidth}\footnotesize \href{{https://kratsg.github.io/cv/?utm_source=cv}}{\faicon{link}~Giordon Stark's Curriculum Vitae}}
%\rfoot{\footnotesize Built \href{https://travis-ci.org/kratsg/cv/builds/\BUILDNUMBER}{\today}\ from \href{https://github.com/kratsg/cv/tree/\COMMITHASH}{\COMMITHASH}}

%%eb\linespread{1.0}

\usepackage{fontawesome5}

\begin{filecontents}{jobname.bib}
@inproceedings{McGoldrick:2016:WCR:2999504.3001118,
    author = {Mc Goldrick, Ciar\'{a}n and Segura, Enrique and Wu, Tianyan and Gerla, Mario},
    title = {WaterCom: Connecting Research Configurations with Practical Deployments: A Multilevel, Multipurpose Underwater Communications Test Platform},
    booktitle = {Proceedings of the 11th ACM International Conference on Underwater Networks \& Systems},
    series = {WUWNet '16},
    year = {2016},
    isbn = {978-1-4503-4637-5},
    location = {Shanghai, China},
    pages = {8:1--8:2},
    articleno = {8},
    numpages = {2},
    url = {http://doi.acm.org/10.1145/2999504.3001118},
    doi = {10.1145/2999504.3001118},
    acmid = {3001118},
    publisher = {ACM},
    address = {New York, NY, USA},
}

@inproceedings{Goldrick:2015:WMM:2831296.2831336,
    author = {Mc Goldrick, Ciar\'{a}n and Matney, Mark and Segura, Enrique and Noh, Youngtae and Gerla, Mario},
    title = {WaterCom: A Multilevel, Multipurpose Underwater Communications Test Platform},
    booktitle = {Proceedings of the 10th International Conference on Underwater Networks \& Systems},
    series = {WUWNET '15},
    year = {2015},
    isbn = {978-1-4503-4036-6},
    location = {Arlington, VA, USA},
    pages = {14:1--14:8},
    articleno = {14},
    numpages = {8},
    url = {http://doi.acm.org/10.1145/2831296.2831336},
    doi = {10.1145/2831296.2831336},
    acmid = {2831336},
    publisher = {ACM},
    address = {New York, NY, USA},
}
\end{filecontents}

\RequirePackage{filecontents}

\begin{filecontents}{jobname1.bib}
@misc{pppl,
    author={Segura Carrillo, Enrique},
    organization = {Princeton Plasma Physics Laboratory Graduate Summer School},
    title ={{Mitigating Slow-Drift Effects for Plasma Wakefield Particle Acceleration}},
    year={2019},
    location = {Princeton, NJ, USA},
}
\end{filecontents}

\usepackage[backend=biber,style=authoryear,   maxcitenames=4, mincitenames=4, 
  maxbibnames=99, minbibnames=99 ]{biblatex}

\usepackage[ 
  maxcitenames=4, mincitenames=4, 
  maxbibnames=99, minbibnames=99 % <-- or some absurd number 
]{biblatex} 

\addbibresource{jobname.bib}
\addbibresource{jobname1.bib}

\usepackage{fancyhdr}
\pagestyle{fancy}
\renewcommand{\headrulewidth}{0pt}
%\input{env}
\fancyhead{}
\fancyfoot{}

%\linespread{1.0}

%______________________________________________________________________________________________________________________
\begin{document}
\nocite{*}
\name{\Large Enrique Alberto Segura Carrillo}

% Center the name over the entire width of resume:
% \moveleft.5\hoffset\centerline{\large\bf \name{ Enrique Alberto Segura Carrillo}
% Draw a horizontal line the whole width of resume:
% \moveleft\hoffset\vbox{\hrule width\resumewidth height 1pt}\smallskip

\begin{resume}

%__________________________________________________________________________________________________________________
% Contact Information
\section{ CONTACT \\ INFORMATION}

Enrique Alberto Segura Carrillo                \hfill \href{mailto:eseguraca.6@gmail.com}{\faIcon{envelope}~eseguraca.6@gmail.com}
\vspace{0mm}\\\vspace{0mm}%
14022 Foothill Blvd Apt 2  \hfill \href{tel:8183100729}{\faIcon{phone}~+1 818-310-0729}%\faIcon{phone}~818-310-0729}
\vspace{0mm}\\\vspace{0mm}%
Sylmar, CA \ \ \ 91342\hfill %\href{https://giordonstark.com/?utm_source=cv}{https://giordonstark.com/}\\
%\vspace{0.5mm}%


%\begin{document}
%\nocite{*}
%\begin{resume}

% we need heading=bibnumbered here to tell biblatex to use \section 
% not \section* (which will produce a spurious * with this class)
%
%\printbibliography[title=PUBLICATIONS,heading=bibnumbered]

% we need heading=bibnumbered here to tell biblatex to use \section 
% not \section* (which will produce a spurious * with this class)
%
% Center the name over the entire width of resume:
 %\moveleft.5\hoffset\centerline{\large\bf Enrique Alberto Segura Carrillo}
% Draw a horizontal line the whole width of resume:
 %\moveleft\hoffset\vbox{\hrule width\resumewidth height 1pt}\smallskip
% address begins here
% Again, the address lines must be centered over entire width of resume: 
%\moveleft.5\hoffset\centerline{eseguraca.6@gmail.com}
%\moveleft.5\hoffset\centerline{(818) 310-0729}
%\moveleft.5\hoffset\centerline{http://linkedin.com/in/eseguraca6/}

\iffalse
\section{OBJECTIVE}
To pursue a Physics PhD in Quantum Information and Computing.
\fi

\section{EDUCATION}
{\sl Master of Science,} Physics - California State University, Los Angeles - June 2021
\\
{\sl Bachelor of Science,} Physics - University of California, Los Angeles - Los Angeles, CA
                      % \sl will be bold italic in New Century Schoolbook (or
	              % any postscript font) and just slanted in
		      %	Computer Modern (default) font 

\section{FELLOWSHIPS}  
Graduate Equity Fellowship \hfill June 2020 \\
NIH MBRS-RISE MS-to-PhD Fellowship \hfill June 2020 \\
Sally Casanova Scholar \hfill May 2020 \\
%HSF Scholarship Finalist \hfill March 2020 \\
SLAC Alonzo W. Ashley Research Fellowship \hfill November 2017 \\
%HSF Procter \& Gamble Orgullosa Scholarship  \hfill January 2013 \\
%AIChE Minority Scholarship Award     \hfill January 2013  \\
%Questbridge College Match Finalist\hfill Fall 2011 \\0

\printbibliography[title=PUBLICATIONS,heading=bibnumbered, type=inproceedings]

\section{RESEARCH} 
{\sl Microelectronics Intern Researcher} \hfill June 2020 - September 2020 \\
The Aerospace Corporation, El Segundo, CA
				\begin{itemize}
				\itemsep -2pt
\item Will develop simulations of physical devices under radiation effects for space applications, implement physics-based model and perform electrical characterization of microelectronic devices to test model's accuracy. 
					\end{itemize}
					

{\sl Graduate Student Researcher} \hfill January 2020 - Present\\
California State University, Los Angeles, Los Angeles CA
				\begin{itemize}
					\itemsep -2pt
					\item Working on experiment automation of Keithley nanovoltmeters, probes and oscilloscopes using LabView and Python to implement a feedback controller to measure spin to spin relaxation time using NMR techniques.
					\end{itemize}


{\sl Visiting Researcher} \hfill November 2018 - November 2019 \\
				UCLA Network Research Laboratory, 
                Los Angeles, CA 
				\begin{itemize}
					\itemsep -2pt
					\item Developed implementation of a Quantum Neural Network (QNN) to study the near term gate quantum computer model using Google's Cirq.  
					\item Tested QNN on a toy-model representing a Stern-Gerlach device under bit-flipping and on 2D Ising model with transverse field.
					\item Designed a pedagogical colab notebook as a tutorial showcasing this implementation.
				\end{itemize}

{\sl Science and Engineering Associate I} \hfill November 2017 - November 2018 \\
				SLAC National Accelerator Laboratory, Menlo Park, CA
				\begin{itemize}
					\itemsep -2pt
					\item Developed Physical Optics Propagation simulations of a Gaussian beam and a flat-top beam propagated on a laser transport comprised of a series of telescopes and mirrors under random perturbations using ZEMAX. 
					\item Created numerical model to predict effects of thermal changes in a laser system intended for Plasma Wake-field Particle acceleration.  
					\item Developed an algorithm by using the detected changes in laser position across the transport, the numerical model, and a PID-controller to stabilize the laser system. 
					\item Designed and built a two-mirrors-two-CCDs system with pico-motors to characterize the effects of hysteresis in the algorithm's performance.
					\item Applied image processing using template matching and low-pass filters to recognize, differentiate, and track the intended laser spot across the laser system. 
					\item Algorithm stabilize the effects of thermal fluctuations by a factor of 40 in 24-hour period. 
					\item Algorithm's feedback enables experiments to run 24/7, dramatically improving from the previous thermally-imposed experimental time-window of 10 pm - 8 am.
					\item Algorithm's control allows uninterrupted experimental runs removing the need for a time-consuming manual alignment every 30 minutes. 
					\item Presented project's results to LCLS team at SLAC to showcase its potential as a suitable feedback method for LCLS's beamline.
				\end{itemize}
				
{\sl Undergraduate Researcher} \hfill Summer 2015 - Summer 2017 \\
                UCLA Network Research Laboratory, 
                Los Angeles, CA 
                 \begin{itemize}  \itemsep -2pt %reduce space between items
                 \item Developed web infrastructure to remotely schedule experiments.
                \item Prepared embedded hardware for experiments by developing parallel compilation of NRL's MPTCP Linux Kernel on Raspberry Pis.
                \item Executed data acquisition of vehicular and underwater wireless experiments.
                \item Implemented data analysis of experimental data to quantify network traffic trade-offs between TCP vs MPTCP.
                \end{itemize}  
                
\section{ADDITIONAL \\ TRAINING}
 
{ Princeton Plasma Physics Laboratory
Graduate Summer School }\hfill  Summer 2019\\
{SLAC Summer Institute }\hfill  Summer 2018 \\
 { 
SLAC Machine Learning Applications for Particle Accelerators Mini-Workshop} \hfill            Spring 2018 \\        

\printbibliography[title=POSTER, type=misc, heading=bibnumbered]     

\section{ TEACHING}  

{\sl Graduate Assistant} \hfill Fall 2019 \\
California State University, Los Angeles, Los Angeles, CA
\begin{itemize} \itemsep -2pt 
\item Supported laboratory physics course focused on experimental debugging to teach physics concepts.
\end{itemize}

{\sl CSL Fellow} \hfill        Fall 2016, Winter 2019  \\
Hispanic Heritage Foundation,   Washington, D.C.
\begin{itemize} \itemsep -2pt 
	\item Developed computer science fundamentals course at Telfair Elementary using Scratch during Winter 2019.
	\item Provided programming fundamentals workshop at San Fernando High School Magnet's AP Computer Science course using Java during Fall 2016.
\end{itemize}

 {\sl Computer Science Instructor} \hfill            Summer 2017 \\
SMASH Academy, University of California, Los Angeles
\begin{itemize}  \itemsep -2pt %reduce space between items
	\item Developed computer science fundamentals course on Python for gifted high school students.
\end{itemize} 

%\iffalse
{\sl Computer Science Instructor} \hfill        Spring 2017  \\
ACM TEACH LA, UCLA Lab School,   Los Angeles, CA
\begin{itemize} \itemsep -2pt 
	\item Co-lead on a computer science course for gifted fifth graders at UCLA Lab School using Python.
\end{itemize} 

%\fi

\iffalse
{\sl CSL Fellow} \hfill        Fall 2016  \\
Hispanic Heritage Foundation,   San Fernando, CA
\begin{itemize} \itemsep -2pt 
	\item Provided programming fundamentals workshop for AP Computer Science students using Java at San Fernando High School Magnet program.
\end{itemize}
\fi

\section{SKILLS}  Python, LabVIEW, MATLAB, Linux, Raspberry Pi 3, Arduino, Fortran, Zemax, Cirq, Qiskit, Mathematica.


\end{resume}

\end{document}